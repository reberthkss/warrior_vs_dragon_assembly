\documentclass[12pt, a4paper]{article}
\usepackage[utf8]{inputenc}
\usepackage[brazil]{babel}
\usepackage{geometry}
\usepackage{graphicx}
\usepackage{listings}
\usepackage{color}
\usepackage{hyperref}

\geometry{left=2.5cm, right=2.5cm, top=2.5cm, bottom=2.5cm}

\title{Guerreiro vs Dragão: Projeto de Jogo em Assembly MIPS}
\author{Bruno Alves - 147938 \\ Reberth Kelvin Santos de Siqueira - 141589 }
\date{\today}

\begin{document}

\maketitle

\begin{abstract}
Este relatório detalha a implementação de "Guerreiro vs Dragão", um jogo de estratégia em turnos desenvolvido como trabalho final para a disciplina de Arquitetura e Organização de Computadores, ministrada pelo Prof. Dr. Fabio Augusto Menocci Cappa no segundo semestre de 2025 na Universidade Federal de São Paulo (UNIFESP). O projeto demonstra conceitos avançados de programação em assembly MIPS, incluindo arquitetura modular, renderização gráfica e lógica de jogo complexa. Uma característica única deste jogo é a mecânica de "Dívida de Juros Compostos", que serve como uma condição de vitória alternativa ao combate tradicional baseado em HP.
\end{abstract}

\tableofcontents
\newpage

\section{Introdução}
"Guerreiro vs Dragão" é um jogo de batalha gráfico em turnos onde o jogador controla um guerreiro lutando contra um dragão. O projeto foi projetado para demonstrar as capacidades da linguagem Assembly MIPS em lidar com lógica, aritmética e E/S mapeada em memória para gráficos.

O objetivo principal é derrotar o dragão reduzindo seus Pontos de Vida (HP) a zero, ou alternativamente, acumular um "Contador de Dívida" de 10.000 através de uma mecânica de juros compostos, efetivamente dominando o inimigo com estratégia econômica.

\section{Arquitetura do Sistema}
O projeto segue uma arquitetura modular para garantir a manutenibilidade e organização do código. A base de código é dividida em vários módulos funcionais:

\begin{itemize}
    \item \textbf{main.asm}: O ponto de entrada da aplicação. Gerencia o loop principal do jogo, verifica as condições de fim de jogo (Vitória/Derrota) e gerencia a lógica de turnos de alto nível.
    \item \textbf{data.asm}: Serve como o repositório central para todas as variáveis de estado do jogo (HP, dívida, contadores de turno), constantes (cores, endereços de memória) e strings de texto (mensagens de UI, perguntas do quiz).
    \item \textbf{macros.asm}: Define macros globais reutilizáveis em todo o projeto, como \texttt{draw\_rectangle} que encapsula a chamada à função de desenho de retângulos, simplificando o código de renderização.
    \item \textbf{battle.asm}: Contém a lógica central de combate. Implementa as funções para os ataques do jogador, comportamento da IA do dragão, cálculos de dano e os algoritmos de juros compostos.
    \item \textbf{quiz.asm}: Implementa um sub-sistema educacional. Gerencia a habilidade ``Conhecimento Arcano'', lidando com a seleção de perguntas, validação de respostas e aplicação de recompensas especiais por respostas corretas.
    \item \textbf{rendering.asm}: O motor gráfico modular. Lida com escritas diretas no display mapeado em memória (0x10040000) para renderizar o céu, chão, sprites (Guerreiro e Dragão), barras de HP e barras de stamina. Inclui funções separadas para renderização de fundo, personagens e interface.
    \item \textbf{sprites.asm}: Armazena os dados gráficos dos personagens do jogo. Cada sprite é definido com um cabeçalho contendo largura e altura, seguido pelos valores de cor RGBA de cada pixel. Inclui onze sprites: \texttt{sprite\_player} (guerreiro em pé), \texttt{sprite\_player\_defeated} (guerreiro derrotado), \texttt{warrior\_shield} (guerreiro com escudo), \texttt{warrior\_spear} (guerreiro lançando lança), \texttt{spear} (projétil lança), \texttt{sprite\_dragon} (dragão padrão), \texttt{sprite\_dragon\_defeated} (dragão derrotado), \texttt{sprite\_dragon\_inferno} (dragão cuspindo fogo), \texttt{sprite\_dragon\_preparing\_inferno} (dragão preparando Inferno), \texttt{sprite\_dragon\_defense} (dragão em postura defensiva) e \texttt{fireball} (projétil bola de fogo). A cor 0x00000000 é tratada como transparente pelo motor de renderização.
\end{itemize}

\subsection{Geração de Sprites}
Para facilitar a criação de sprites, foi desenvolvido o script Python \texttt{sprites/converter\_sprites.py} que converte imagens PNG em formato compatível com MIPS Assembly. O conversor redimensiona as imagens para a largura desejada, extrai os valores RGB de cada pixel e gera automaticamente o código Assembly no formato \texttt{.word} adequado para inclusão no arquivo \texttt{sprites.asm}.

\section{Personagens}

O jogo apresenta dois personagens principais em um confronto épico:

\subsection{O Guerreiro (Jogador)}
O protagonista é um guerreiro medieval controlado pelo jogador. Ele possui:
\begin{itemize}
    \item \textbf{HP Inicial}: 100 pontos de vida
    \item \textbf{Stamina Inicial}: 100 pontos (regenera 15 por turno)
    \item \textbf{Posição no Display}: Lado esquerdo da tela (X=50, Y=185)
    \item \textbf{Sprites}: Múltiplas poses incluindo postura normal, com escudo, lançando lança e derrotado
    \item \textbf{Recursos}: 2 Estus Flasks para recuperação de HP
    \item \textbf{Habilidades}: 7 ações disponíveis (Pular Turno, Escudo, Espada, Flanco, Lança, Conhecimento Arcano e Estus Flask)
\end{itemize}

\subsection{O Dragão (Inimigo)}
O antagonista é um poderoso dragão controlado pela IA do jogo. Ele possui:
\begin{itemize}
    \item \textbf{HP Inicial}: 1000 pontos de vida (compensando sua baixa taxa de acerto)
    \item \textbf{Stamina Inicial}: 100 pontos (regenera 15 por turno)
    \item \textbf{Posição no Display}: Lado direito da tela (X=180, Y=185), podendo voar para Y=140
    \item \textbf{Sprites}: Múltiplas poses incluindo postura normal, voando, cuspindo fogo, preparando Inferno, em defesa e derrotado
    \item \textbf{Comportamento}: Seleção aleatória entre 4 ataques (25\% cada), com validação de stamina
    \item \textbf{Ataques}: Sopro de Fogo (20 stamina), Pisar (30 stamina), Voar (25 stamina) e Inferno (50 stamina)
    \item \textbf{Fallback}: Quando sem stamina suficiente, entra em postura de defesa
\end{itemize}

\section{Mecânicas de Jogo}

\subsection{Apresentação de Informações}
O jogo opera em um sistema de turnos alternados entre o jogador e o dragão. Para acompanhar a batalha completamente, é necessário observar duas interfaces simultaneamente:

\textbf{Console (Run I/O):}
\begin{itemize}
    \item Exibe o menu de ações disponíveis para o jogador
    \item Mostra mensagens de ataque, dano causado e efeitos especiais
    \item Apresenta o status da batalha (HP do jogador, HP do dragão, contador de dívida)
    \item Recebe a entrada do jogador (números 1-6 para selecionar ações)
    \item Exibe as perguntas do quiz e suas opções de resposta
\end{itemize}

\textbf{Bitmap Display:}
\begin{itemize}
    \item Renderiza os sprites do guerreiro e do dragão
    \item Mostra o cenário (céu azul e chão verde)
    \item Exibe as barras de HP de ambos os personagens
    \item Indica visualmente o turno atual através de um cursor amarelo
    \item Mostra o dragão em posição elevada quando está voando
    \item Exibe sprites de derrota quando um personagem é derrotado
\end{itemize}

Esta combinação de saída textual e gráfica proporciona uma experiência completa, onde o console fornece informações detalhadas sobre as mecânicas do jogo enquanto o display visual oferece feedback imediato sobre o estado da batalha.

\subsection{Sistema de Combate}
O combate é baseado em turnos. O jogador tem acesso a sete ações distintas, cada uma com custo de stamina específico:
\begin{enumerate}
    \setcounter{enumi}{-1}
    \item \textbf{Pular Turno (Skip)}: O jogador não realiza nenhuma ação, útil para regenerar stamina. Custo: 0 stamina.
    \item \textbf{Escudo (Shield)}: Ativa um escudo que absorve 50 HP de dano. Enquanto o escudo está ativo, ações ofensivas (2-5) são bloqueadas. Use novamente para cancelar. Custo: 0 stamina.
    \item \textbf{Espada (Sword)}: Um movimento tático que atordoa o dragão, fazendo-o perder um turno, e aplica juros compostos à dívida. Não causa dano direto. Custo: 25 stamina.
    \item \textbf{Flanco (Flank)}: Um ataque de alto risco e alta recompensa com 40\% de chance de acerto crítico e dano aumentado (15-24 HP, 30 crítico). Taxa de acerto: 80\%. Custo: 40 stamina.
    \item \textbf{Lança (Spear)}: Ataque animado com projétil que causa menos dano (5-9 HP, 15 crítico), mas aumenta a evasão do jogador para o próximo turno. Taxa de acerto: 80\%. Custo: 20 stamina.
    \item \textbf{Conhecimento Arcano (Arcane Knowledge)}: Uma habilidade especial que aciona uma pergunta sobre arquitetura de computadores. Resposta correta aplica 5x juros compostos; resposta errada causa -5 HP de penalidade (absorvido pelo escudo se ativo). Custo: 50 stamina.
    \item \textbf{Estus Flask}: Um item consumível inspirado na icônica mecânica de recuperação de HP do jogo \textit{Dark Souls}. O jogador começa com 2 frascos disponíveis. Ao usar, regenera 25 HP imediatamente e mais 25 HP por turno durante 2 turnos adicionais. Custo: 0 stamina.
\end{enumerate}

\begin{table}[h]
\centering
\caption{Habilidades do Jogador}
\begin{tabular}{|l|c|c|c|c|c|}
\hline
\textbf{Ação} & \textbf{Stamina} & \textbf{Acerto} & \textbf{Acerto (Voando)} & \textbf{Crítico} & \textbf{Dano/Efeito} \\
\hline
Pular & 0 & -- & -- & -- & Nenhum \\
\hline
Escudo & 0 & -- & -- & -- & Absorve 50 HP \\
\hline
Espada & 25 & 100\% & 100\% & -- & Atordoa dragão \\
\hline
Flanco & 40 & 80\% & 50\% & 40\% & 15-24 (30 crit) \\
\hline
Lança & 20 & 80\% & 50\% & 15\% & 5-9 (15 crit) + Evasão \\
\hline
C. Arcano & 50 & -- & -- & -- & 5x juros / -5 HP \\
\hline
Estus & 0 & -- & -- & -- & +25 HP/turno (3x) \\
\hline
\end{tabular}
\end{table}

O Dragão atua como o oponente de IA. A cada turno, seleciona aleatoriamente entre quatro ataques (25\% cada), mas deve ter stamina suficiente para executá-los. Se não houver stamina para nenhum ataque, entra em postura de defesa.
\begin{itemize}
    \item \textbf{Sopro de Fogo} (20 stamina): Ataque animado com projétil de bola de fogo. Dano de 25-40 HP (60 crítico). Taxa de acerto: 35\% (50\% se jogador tiver evasão) com 15\% de chance de crítico. Reduz a dívida em 5\% ao acertar.
    \item \textbf{Pisar (Stomp)} (30 stamina): Atordoa o jogador, fazendo-o perder um turno. Reduz a dívida em 5\%. Sem dano direto.
    \item \textbf{Voar (Fly)} (25 stamina): Aumenta a evasão do dragão. O jogador precisa de 50+ para acertar no próximo turno (reduz acerto de 80\% para 50\%). O dragão é renderizado em posição elevada (Y=140).
    \item \textbf{Inferno} (50 stamina): Ataque em duas fases. No primeiro turno, o dragão ``acumula fogo'' (exibe sprite de preparação). No turno seguinte, desencadeia o ataque devastador com 80\% de acerto e 45-65 HP de dano. Reduz a dívida em 5\% ao acertar.
    \item \textbf{Postura de Defesa} (0 stamina): Fallback automático quando o dragão não tem stamina para nenhum ataque. Exibe sprite de defesa e aguarda regeneração de stamina.
\end{itemize}

\begin{table}[h]
\centering
\caption{Habilidades do Dragão}
\begin{tabular}{|l|c|c|c|c|c|}
\hline
\textbf{Ataque} & \textbf{Stamina} & \textbf{Acerto} & \textbf{Acerto (Evasivo)} & \textbf{Crítico} & \textbf{Dano/Efeito} \\
\hline
Sopro de Fogo & 20 & 35\% & 50\% & 15\% & 25-40 (60 crit) \\
\hline
Pisar & 30 & 100\% & 100\% & -- & Atordoa jogador \\
\hline
Voar & 25 & 100\% & 100\% & -- & +Evasão dragão \\
\hline
Inferno & 50 & 80\% & 80\% & -- & 45-65 (2 turnos) \\
\hline
Defesa & 0 & -- & -- & -- & Aguarda stamina \\
\hline
\end{tabular}
\end{table}

\subsection{Sistema de Stamina}
Uma mecânica de gerenciamento de recursos que adiciona profundidade estratégica ao combate:
\begin{itemize}
    \item \textbf{Stamina Máxima}: Tanto o jogador quanto o dragão possuem 100 pontos de stamina.
    \item \textbf{Regeneração}: A cada início de turno, 15 pontos de stamina são regenerados automaticamente (não excede 100).
    \item \textbf{Custo de Habilidades}: Cada habilidade possui um custo específico de stamina. Se o personagem não tiver stamina suficiente, a ação falha e exibe uma mensagem de erro.
    \item \textbf{Custos do Jogador}: Pular/Escudo/Estus (0), Lança (20), Espada (25), Flanco (40), Conhecimento Arcano (50).
    \item \textbf{Custos do Dragão}: Sopro de Fogo (20), Voar (25), Pisar (30), Inferno (50).
    \item \textbf{Barra Visual}: Barras de stamina azuis são exibidas abaixo das barras de HP no display gráfico.
    \item \textbf{Fallback do Dragão}: Quando o dragão não possui stamina para nenhum ataque, automaticamente entra em postura de defesa.
\end{itemize}

\subsection{Sistema de Dívida de Juros Compostos}
Uma mecânica única envolvendo um ``Contador de Dívida''.
\begin{itemize}
    \item \textbf{Crescimento}: Cada vez que o jogador acerta um golpe, o contador de dívida cresce 10\% mais um valor base de 100.
    \item \textbf{Redução}: Quando o dragão atinge o jogador, a dívida é reduzida em 5\%, simulando um revés.
    \item \textbf{Vitória}: Se o contador de dívida atingir 10.000, o jogador vence imediatamente via ``Vitória por Juros Compostos''.
    \item \textbf{Inspiração}: Esta mecânica foi inspirada nas habilidades do personagem Knuckle Bine, do anime \textit{Hunter x Hunter}, onde o acúmulo de ``juros'' de aura leva à derrota do oponente.
\end{itemize}

\subsection{Conhecimento Arcano (Sistema Educacional)}
O jogo integra conteúdo educacional diretamente na jogabilidade. A ação ``Conhecimento Arcano'' apresenta perguntas aleatórias sobre Arquitetura de Computadores (ex: sobre ULA, RAM, Barramentos).
\begin{itemize}
    \item \textbf{Resposta Correta}: Aplica a fórmula de juros compostos 5 vezes instantaneamente, fornecendo um grande impulso para a condição de vitória por dívida.
    \item \textbf{Resposta Errada}: Penaliza o jogador com perda de HP.
\end{itemize}

\subsection{Condições de Vitória e Derrota}
O jogo apresenta múltiplas formas de conclusão da batalha:

\textbf{Vitória do Jogador:}
\begin{itemize}
    \item \textbf{Vitória por HP}: Reduzir o HP do dragão a zero ou menos através de ataques diretos.
    \item \textbf{Vitória por Juros Compostos}: Acumular o contador de dívida até atingir 10.000 pontos. Esta é a vitória estratégica que recompensa jogadores que conseguem manter pressão constante enquanto evitam dano.
\end{itemize}

\textbf{Derrota do Jogador:}
\begin{itemize}
    \item \textbf{Derrota por HP}: O jogador perde quando seu HP chega a zero ou menos. Neste momento, o guerreiro é exibido em sua sprite de derrota (deitado no chão) e a barra de HP muda para vermelho.
\end{itemize}


\section{Implementação Técnica}

\subsection{Motor Gráfico}
O jogo roda em um display de 256x256 pixels com profundidade de cor de 32 bits. O módulo \texttt{rendering.asm} usa endereçamento de memória eficiente para desenhar os pixels.
\begin{itemize}
    \item \textbf{Cálculo de Endereço}: $Base + (Y \times 256 + X) \times 4$. A multiplicação por 256 é otimizada usando um deslocamento lógico à esquerda (\texttt{sll}) de 8 bits.
    \item \textbf{Sprites}: Sprites são armazenados com cabeçalhos de largura e altura, e o loop de renderização ignora a cor de transparência (0x00000000) para sobrepor os personagens no fundo.
\end{itemize}

\subsection{Geração de Números Aleatórios}
O jogo utiliza extensivamente a chamada de sistema (syscall) 42 para gerar números aleatórios para determinar:
\begin{itemize}
    \item Sucesso do ataque (cálculos de Acerto/Erro).
    \item Acertos críticos.
    \item Escolhas da IA do Dragão.
    \item Seleção de perguntas do Conhecimento Arcano.
\end{itemize}

\subsection{Ambiente de Teste}
O projeto foi desenvolvido e validado utilizando o simulador MARS (MIPS Assembler and Runtime Simulator). Para a saída gráfica, foi utilizada a ferramenta \textit{Bitmap Display} incluída no simulador, com as seguintes configurações específicas para garantir a visualização correta:

\begin{itemize}
    \item \textbf{Unit Width in Pixels}: 1
    \item \textbf{Unit Height in Pixels}: 1
    \item \textbf{Display Width in Pixels}: 256
    \item \textbf{Display Height in Pixels}: 256
    \item \textbf{Base address for display}: 0x10040000 (heap)
\end{itemize}

\section{Desafios e Decisões de Projeto}
Durante o desenvolvimento do projeto, foram identificados desafios técnicos significativos, especialmente relacionados à renderização gráfica. Um dos principais obstáculos foi o desenho do \textit{bitmap}, onde inicialmente tentou-se utilizar um endereço de memória estático para a manipulação dos pixels. No entanto, para o correto funcionamento com a ferramenta de display gráfico do simulador, deveríamos ter utilizado o endereço de memória da \textit{heap} (dinâmica). Essa divergência causou dificuldades iniciais na exibição correta das sprites e cores na tela, exigindo uma refatoração do código de renderização para apontar para o endereço base correto (0x10040000).

Além disso, devido à complexidade inerente ao desenvolvimento em baixo nível com Assembly, foi tomada a decisão de priorizar a profundidade e robustez das mecânicas de jogo — como o sistema de combate, as perguntas do quiz e o cálculo de juros compostos — em detrimento de uma apresentação visual mais elaborada. O foco principal foi garantir que a lógica do jogo funcionasse perfeitamente, mantendo os gráficos funcionais, porém simples, para assegurar a entrega de um sistema estável e livre de bugs críticos.

\section{Conclusão}
O projeto "Guerreiro vs Dragão" cria com sucesso um jogo RPG envolvente usando linguagem Assembly de baixo nível. Ele demonstra que lógica complexa, design de software modular e interfaces gráficas podem ser efetivamente implementados mesmo sem abstrações de alto nível, fornecendo insights profundos sobre arquitetura de computadores e operações em nível de máquina.

\end{document}
